\normalfont\documentclass[letterpaper,11pt]{article}
\usepackage{amsmath, amsfonts,amssymb,latexsym}
\usepackage{fullpage}
\usepackage{parskip}
\usepackage{flexisym}
\usepackage{indentfirst}
\usepackage{algorithm}
\usepackage{algorithmicx}
\usepackage{hyperref}
\usepackage{algpseudocode}
\begin{document}
\setlength{\parindent}{2ex}
\newcommand{\header}{
	\noindent \fbox{
	\begin{minipage}{6.4in}
	\medskip
	\textbf{CS 260 Fundamentals of the Design and Analysis of Algorithms} \hfill \textbf{Fall 2016} \\[1mm]
	\begin{center}
		{\Large HomeWork 4} \\[3mm]
	\end{center}
	  Name: \itshape{Liangjian Chen} \\
	  \textnormal{ID}: \itshape{\#52006933} \hfill \today
	\medskip
	\end{minipage}}
}
\bigskip
\header

\begin{enumerate}
\item [Problem 5.1]\textbf{Solution:}\par
	Assume that $A(i)$ means the $i^{th}$ smallest number in Database $A$, and $B(i)$ means the $i^{th}$ smallest number in Database $B$.\par
	\begin{algorithmic}
		\Function {Find\_Median}{$n, L_A, L_B$}
			\If{$N == 1$}
				\State \Return Min($A(L_A + 1), B(L_B + 1)$)
			\EndIf
			\State $mid = \lceil \frac{n}{2} \rceil$
			\If {$A(L_A + mid) \le B(L_B + mid)$}
				\State \Call{Find\_Median}{$mid , L_A + \lfloor \frac{n}{2} \rfloor, L_B$}
			\Else
				\State \Call{Find\_Median}{$mid , L_A , L_B + \lfloor \frac{n}{2} \rfloor$}
			\EndIf
		\EndFunction
	\end{algorithmic}
	Initially, call function \Call{Find\_Median}{$n , 0, 0$} \par
	Every time, we call two median values in both database($mid = \lceil \frac{n}{2} \rceil$), we can see if $A(mid) \le B(mid)$, then $A(1)..A(\lfloor \frac{n}{2} \rfloor)$ would not be the answer, $B(\lceil \frac{n}{2} \rceil)..B(N)$ would not be the answer. So, every time we eliminate half of the possible answer which leads the recurrence $T(n) = T(n / 2) + O(1)$, solve this recurrence,we get $T(n) = O(log(n))$
\item [Problem 5.3]\textbf{Solution:}\par
\item [Problem 5.5]\textbf{Solution:}\par

\end{enumerate}
\end{document}
