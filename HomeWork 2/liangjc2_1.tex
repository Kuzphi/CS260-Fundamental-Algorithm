\normalfont\documentclass[letterpaper,11pt]{article}
\usepackage{amsmath, amsfonts,amssymb,latexsym}
\usepackage{fullpage}
\usepackage{parskip}
\usepackage{flexisym}
\usepackage{algorithm}
\usepackage{indentfirst}
\usepackage{graphicx}
\usepackage{algorithmicx}
\usepackage{algpseudocode}
\begin{document}
\setlength{\parindent}{2ex}
\newcommand{\header}{
	\noindent \fbox{
	\begin{minipage}{6.4in}
  	\medskip
  	\textbf{CS 260 - Fundamentals of the Design and Analysis of Algorithms} \hfill \textbf{Fall 2016} \\[1mm]
  	\begin{center}
    	{\Large HomeWork 2} \\[3mm]
  	\end{center}
	\today \hfill \itshape{Liangjian Chen}
	\medskip
	\end{minipage}}
}

\bigskip
\header

\begin{enumerate}
\item (Problem 2)\par
	$10^{10} * 60 * 60 = 3.6 * 10^{13}$
	\begin{enumerate}
		\item
			$n = \sqrt{3.6*10^{13}}= 6 * 10^6$
		\item
			$n = \sqrt[3]{3.6*10^{13}} = 33019$
		\item
			$n = 6 * 10^5$
		\item
			$n \approx 1.4 * 10^{11}$(when base = e)
		\item
			$n = 45$
		\item
			$n = 5$
	\end{enumerate}
\item (Problem 4)\par
	order is $g_1, g_5,g_3,g_4,g_2,g_7,g_6$.
\item (Problem 5)\par
	\begin{enumerate}
		\item False, $f(n) = 10, g(n) = 1$
		\item False, $f(n) = n^2 + n, g(n) = n^2$
		\item True,\\
		 $\forall n \geq n_0, f(n) \leq c * g(n) \\ 
		  \forall n \geq n_0, f(n)^2 \leq c^2 * g(n)^2\\
		  f(n)^2 = O(g(n)^2)
		 $
	\end{enumerate}
\item (Problem 6) $f(n)$ can be calculated directly.
\begin{flalign*}
f(n) &= \sum_{i = 1}^n\sum_{j = i}^n(j - i + 1)& \\
	 &= \frac{1}{2}\sum_{i = 1}^n(n-i+2)*(n-i+1)&\\
	 &= \frac{1}{2}\sum_{i = 1}^n i(i + 1)&\\
	 &= \frac{n(n+1)(2n + 1)}{12}+ \frac{n(n + 1)}{4}&\\
	 &= \frac{n(n+1)(n + 2)}{6}&
\end{flalign*}
\begin{enumerate}
	\item $O(f(n)) = n^3$
	\item $\Omega(f(n)) = n^3$
	\item Original algorithm makes a lot of redundant work. Instead of iterating from $a[i]$ to $a[j]$, $B[i,j]$ could be simply transfered from either $B[i,j-1]$ or $B[i,j+1]$.
	\begin{algorithm}
	\caption{Improved Algorithm:}
	\begin{algorithmic}[1]
		\For{$i = 1,2,...,n $}
		\State $B[i,i] \gets A[i]$
			\For{$j = i - 1, i - 2, ..., 1$}
				\State $B[i,j] \gets B[i,j+1] + A[j]$
			\EndFor
			\For{$j = i + 1,i + 2,...,n$}
				\State $B[i,j] \gets B[i,j-1] + A[j]$
			\EndFor
		\EndFor
	\end{algorithmic}
	\end{algorithm}\par
	In this algorithm,
	\begin{flalign*}
	g(n) &= \sum_{i = 1}^n{(1 + \sum_{j = 1}^{i - 1}{1} + \sum_{j = i + 1}^n {1}})& \\
		 &= \sum_{i = 1}^n n&\\
		 &= \frac{n(n+1)}{2}&\\
		 &= \Theta(n^2)&
	\end{flalign*}

\end{enumerate}
\end{enumerate}
\end{document}
