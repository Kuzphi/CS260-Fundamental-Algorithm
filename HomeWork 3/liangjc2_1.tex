\normalfont\documentclass[letterpaper,11pt]{article}
\usepackage{amsmath, amsfonts,amssymb,latexsym}
\usepackage{fullpage}
\usepackage{parskip}
\usepackage{flexisym}
\usepackage{indentfirst}
\usepackage{algorithm}
\usepackage{algorithmicx}
\usepackage{hyperref}
\usepackage{algpseudocode}
\begin{document}
\setlength{\parindent}{2ex}
\newcommand{\header}{
	\noindent \fbox{
	\begin{minipage}{6.4in}
  	\medskip
  	\textbf{CS 260 - Fundamental Algorithm} \hfill \textbf{Fall 2016} \\[1mm]
  	\begin{center}
    	{\Large HomeWork 1} \\[3mm]
  	\end{center}
	\today \hfill \itshape{Liangjian Chen}
	\medskip
	\end{minipage}}
}

\bigskip
\header


	

\begin{enumerate}
\item (Problem 2)\par
\begin{algorithm*}
\caption{Find a circle in a graph}
\begin{algorithmic}
	\Function {Find\_Circle}{$u, e^\prime$}
		\State push $u$ into stack
		\For{$e(u,v) \in E \text{ except } e^\prime$}
			\If{$v\text{ is in the stack}$}
				\State report $v$ is in the circle
				\While{the top of stack($x$) is not $v$}
					\State report $x$ is in the circle
					\State pop $x$
				\EndWhile
				\State \Return find a circle
			\ElsIf{$v \text{ haven't been visited}$}
				\If {\Call{Find\_Circle}{$v, e(u,v)$} finds a circle}				
					\State \Return find a circle
				\EndIf
			\EndIf
		\EndFor
		\State \Return do not find a circle
	\EndFunction
	\State Initialize: make a stack empty
	\State \Call{Find\_Circle}{1,None}
\end{algorithmic}
\end{algorithm*}
\item (Problem 3)\par
First, apply topological ordering algorithm into the graph. If algorithm delete all the node in the graph, it is a DAG. Otherwise, apply BFS into the rest of graph to find connected components. Every connected component is a circle. Both topological ordering and BFS is $O(m + n)$.
\item (Problem 4)\par
It is a typical problem about disjoint set with distance. Let's define two butterflies is in same specie, if the distance between these two are even. Otherwise, the distance is odd. Like typical disjoint set, every butterfly is a set only consist itself at begining. The algorithm going through all m judgments and merge two set together. However instead of merging them directly, here we merge two sets by carefully choosing the distance($1$ or $0$) between two representatives to meet our definition(Odd or Even). When two butterflies are in the same set, we are able to check its consistency by checking their distance.\par
initialize disjoint set is $O(n)$. There are $O(m)$ disjoint set operations. Each of them costs $O(1)$. Total time is $O(n + m)$.\\
There is a same question in \url{http://poj.org/problem?id=2492}
\item (Problem 6)\par
Assume that $T$'s edges set is $E_T$, then for some extra set $E_e$, $E = E_T \cup E_e, E_T \cap E_e = \varnothing$. Now I want to argue that $E_e = \varnothing$. Assume that there is one edge $e(a,b)$ in set $E_e$, and define $d(x,y)$ as the distance between two any point $x$, $y$, which is the number of edges that the path between $a$,$b$ contains in the tree $T$. Because $T$ is BFS tree, so $|d(u,a) - d(u,b)| \leq 1$. Also, $T$ is a DFS tree, so, one of $a$,$b$ must be a ancestor of another. However, if both $|d(u,a) - d(u,b)| \leq 1$ and condition of ancestor hold, $e(a,b)$ must in the set $E_T$, it contradicts with assumption. So $E_e = \varnothing$.
\item (Problem 9)\par
	First, we can use BFS starting from $u$ to find a path between $u$ and $v$. Then, for every node $x$, record $d(x)$ which is the distance between $x$ and $u$ in BFS. We group the node by their $d(x)$, then among all groups, at least one group only has one node (Otherwise the total number of node would be larger than $n$). Then this node is the key node.(delete it there is not path between $u$ and $v$). There is just a BFS in algorithm which cost $O(n + m)$.
\item (Problem 10)\par
This problem could be solved by dynamic programming. Let $dp[x]$ represent the number of shortest path from $u$ to $x$. Trans-equation is $$dp[x] = \sum_{\text{exist shortest paths } u \to..\to y \to x}{dp[y]}$$\\ Luckily, in this graph, every weight of edge is $1$. So a BFS search framework can support this dynamic programming procedure.
\begin{algorithm*}
\caption{Find the number of different shortest path}
\begin{algorithmic}
	\Require a Graph $G(V,E)$,two nodes $v,w$ .
	\State make queue $Q$ empty
	\State initialize all element in array $dp$ zero.
	\State put node $v$ into $Q$
	\While{$Q$ is not empty}
		\State $x \gets$ the front of queue
		\State pop the front of queue
		\For { $\forall e(x,y) \in E$}
			\If{$y$ has not visited}
				\State $dp[y] \gets dp[y] + dp[x]$
				\State push $y$ into $Q$	
			\EndIf
		\EndFor
	\EndWhile 
	\State \Return $dp[w]$
\end{algorithmic}
\end{algorithm*}\par
My algorithm goes through every node and every edges $O(1)$ time, so time complexity is $O(n + m)$.
\item (Problem 12)\par
It is a problem of difference constraints. Let's assign two number to every single people($P_i$), $B_i$ and $D_i$ which represent his/her time of birth and time of dead. Then we have three type of different difference constrain.
\begin{enumerate}
	\item[(1)]
	The dead time should be later than birth time for everyone. So first set of difference constraints, $D_i - B_i > 0$, which means $B_i - D_i \leq -1$.
	\item[(2)]
	For every constrain of kind of $P_i$ died before $P_j$. The constrain is $B_i > D_j$ which is $D_j - B_i \leq -1$.
	\item[(3)]
	For every pair of $P_i,P_j$ overlapping, the constrain is $B_i < D_j$ and $B_j < D_i$. which is $B_i - D_j \leq -1$ and $B_j - D_i \leq -1$.
\end{enumerate}
Recall the triangle inequality in shortest path problem, $d[v] - d[u] \leq c$ holds, if a edge from $u$, to $v$, cost $c$ exist. Base on triangle inequality, we could build a graph by add a $c$-cost edge between $u$,$v$ if we have a inequality $u - v \leq c$. Then use SPFA algorithm or Bellman-Ford algorithm to check whether there is a negative loop in this graph (In this problem, since all weights are negative, we could simply check whether there is a cycle instead of a negative circle). If there is, this system is not consist. Otherwise, we could assign arbitrary number to starting node, then inference all other number by the shortest distance between that node and starting node.
\end{enumerate}
\end{document}
