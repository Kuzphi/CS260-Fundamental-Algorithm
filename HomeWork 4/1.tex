\normalfont\documentclass[letterpaper,11pt]{article}
\usepackage{amsmath, amsfonts,amssymb,latexsym}
\usepackage{fullpage}
\usepackage{parskip}
\usepackage{flexisym}
\usepackage{indentfirst}
\usepackage{algorithm}
\usepackage{algorithmicx}
\usepackage{hyperref}
\usepackage{algpseudocode}
\usepackage{pythonhighlight}
\begin{document}
\setlength{\parindent}{2ex}
\newcommand{\header}{
	\noindent \fbox{
	\begin{minipage}{6.4in}
	\medskip
	\textbf{CS 260 Fundamentals of the Design and Analysis of Algorithms} \hfill \textbf{Fall 2016} \\[1mm]
	\begin{center}
		{\Large HomeWork 2} \\[3mm]
	\end{center}
	  Name: \itshape{Liangjian Chen} \\
	  \textnormal{ID}: \itshape{\#52006933} \hfill \today
	\medskip
	\end{minipage}}
}
\bigskip
\header


	

\begin{enumerate}
\item [Problem 3]\textbf{Solution:}\par
let's assume that array $A$ is the array of all package's weight stored by their coming order. Function $MinTruck(X)$ is the minimum number of truck to send all package in $A$. Then an apparent truth is that $MinTruck(A[a:]) \le MinTruck(A[b:])$ if $b \le a$ ($A[a:]$ means the $A$'s subarray which its index from $a$ to the end). 
	The proof is consider the best arrangement of $MinTruck(A[b:])$. First subtract the item in $A[a:b]$, after that if a truck is empty then discard it. Now we get an possible arrangement of $A[a:]$. And because we  arrangement less than 
\item [Problem 4]\textbf{Solution:}\par
\begin{python}
def Subsequence(S, S_prime):
	j = 0
	for event in S:
		if S_prime[j] == event:
			j += 1
		if j == len[S_prime]:
			break
	if j == len[S_prime]:
		return "Yes"
	return "No"
\end{python}
\item [Problem 8]\textbf{Solution:}\par
Assume that $T$ and $T^\prime$ is two different minimum spanning trees. An edge $e \in T$,and $e\not\in T^\prime$. Let's add $e$ into Tree $T$, then there are two different situation.
\begin{enumerate}
	\item$e$ is the most expensive among the circle\\
		Because all weights are different, at least one edge's weight larger then $e$, let's assume it is $e^\prime$. Substitute $e$ for $e^\prime$, it forms a new tree with smaller weight. Contradict with $T^\prime$ is a minimum spanning tree.
	\item$e$ is not the most expensive among the circle\\
		There must be an edge in the circle which is not belong to tree $T$(Otherwise $e$ form a circle in the tree $T$).Let assume it is $e^{\prime\prime}$. By substituting $e^{\prime\prime}$ for $e$ in tree $T$, we construct a new tree with smaller sum weight. It contradicts with the assumption.
\end{enumerate}
In conclusion, $G$'s minimum spanning tree is unique.
\item [Problem 9]\textbf{Solution:}\par
\begin{enumerate}
	\item
	\item
\end{enumerate}
\item [Problem 11]\textbf{Solution:}\par
\item [Problem 17]\textbf{Solution:}\par
\item [Problem 27]\textbf{Solution:}\par
\end{enumerate}
\end{document}
