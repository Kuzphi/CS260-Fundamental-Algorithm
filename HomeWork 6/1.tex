\normalfont\documentclass[letterpaper,11pt]{article}
\usepackage{amsmath, amsfonts,amssymb,latexsym}
\usepackage{fullpage}
\usepackage{parskip}
\usepackage{flexisym}
\usepackage{indentfirst}
\usepackage{algorithm}
\usepackage{algorithmicx}
\usepackage{hyperref}
\usepackage{algpseudocode}
\begin{document}
\setlength{\parindent}{2ex}
\newcommand{\header}{
	\noindent \fbox{
	\begin{minipage}{6.4in}
	\medskip
	\textbf{CS 260 Fundamentals of the Design and Analysis of Algorithms} \hfill \textbf{Fall 2016} \\[1mm]
	\begin{center}
		{\Large HomeWork 6} \\[3mm]
	\end{center}
	  Name: \itshape{Liangjian Chen} \\
	  \textnormal{ID}: \itshape{\#52006933} \hfill \today
	\medskip
	\end{minipage}}
}
\bigskip
\header

\begin{enumerate}
\item [Problem 1]\textbf{Solution:}\par
	\begin{enumerate}
		\item 99 - 100 - 99
		\item 100 - 1 - 1 - 100 
		\item Denote $f[i][0]$ represents the maximum result of among first $i$ nodes, node $i$ is not chosen. $f[i][1]$ represents the maximum result of among first $i$ nodes, node $i$ is chosen. Then 

		$$ f[i][j]=\left\{
			\begin{array}{ll}
			0   &    i = 0\\
			max(f[i-1][1],f[i-1][0]) &           i > 0 \text{ and } j=0\\
			f[i-1][0] + w_i  &          i > 0 \text{ and } j = 1\\
			\end{array} \right. 
		$$
		Final result is $\max\{f[n][0],f[n][1]\}$.\par
		The state space is $O(n)$, transition function is $O(1)$. Total time is $O(n)$.
	\end{enumerate}
		
\item [Problem 2]\textbf{Solution:}\par
	\begin{enumerate}
		\item $\ell_1 = 0, \ell_2 = 100, \ell_3 = 1000$,$h_1 = 10, h_2 = 10, h_3 = 10$
		Correct answer is choose $h_1$ and $\ell_3$ whose sum is $1010$. The algorithm will give $\ell_2$ and $h_3$, whose sum is $110$.
		\item Denote $f[i][0]$ represents the maximum result of among first $i$ weeks, state week $i$ is None. $f[i][1]$ represents the maximum result of among first $i$ nodes, state of week $i$ Occupied(either low or high stress). Then 
		$$ f[i][j]=\left\{
			\begin{array}{ll}
			0   &    i = 0\\
			max(f[i-1][1],f[i-1][0]) &           i > 0 \text{ and } j=0\\
			max(f[i-1][0] + h_i, f[i-1][1] + \ell_i)  &          i > 0 \text{ and } j = 1\\
			\end{array} \right. 
		$$
		Final result is $\max\{f[n][0],f[n][1]\}$.\par
		The state space is $O(n)$, transition function is $O(1)$. Total time is $O(n)$.
	\end{enumerate}
\item [Problem 3]\textbf{Solution:}\par
\begin{enumerate}
		\item 6 nodes, 6 edges. (1,2),(2,5),(1,3),(3,4),(4,5),(5,6).
		Correct answer is $1\to 3 \to 4\to 5 \to 6$. Algorithm would find $1\to 2 \to 5$.
		\item find the topological ordering of this graph stored in array $A$. Denote $f[i]$ as the longest path end at $A[i]$.
		$$ f[i][j]=\left\{
			\begin{array}{ll}
			0   &    \text{no incoming edge} \\
			max(f[j]) + 1 &           \forall edge (j,A[i])\\
			\end{array} \right. 
		$$
		Final result is $f[n]$.\par
		The state space is $O(n)$, transition function is $O(1)$. Find the topological ordering is $O(n)$. Thus total time is $O(n)$.
	\end{enumerate}
\item [Problem 4]\textbf{Solution:}\par
	\begin{enumerate}
		\item $M = 100$,$N_1 = 1, N_2 = 10, N_3 = 10$,$h_1 = 10, h_2 = 1, h_3 = 1$.
		Correct answer is $12$. The algorithm will give $103$.
		\item  $M = 0$, costs are given by the following table.\\
		\begin{tabular}{|c|c|c|c|c|}
			\hline
			&Month 1& Month 2 & Month 3 & Month 4   \\
			\hline
			NY & 1000 & 0 & 1000 & 0 \\
			\hline
			SF & 0 & 1000 & 0 & 1000 \\
			\hline
		\end{tabular}\\
		Since the transport fee is 0, the we would choose the lower price every month which means locationmust be changed every month.
		\item Denote $f[i][0]$ represents the maximum result of among first $i$ month, during month $i$ you work in NY. $f[i][1]$ represents the maximum result of among first $i$ month, during month $i$ you work in SF.
		$$ f[i][j]=\left\{
			\begin{array}{ll}
			0   &    i = 0\\
			max(f[i-1][1] + N_i + w, f[i-1][0] + N_i) &           i > 0 \text{ and } j = 0\\
			max(f[i-1][1] + S_i, f[i-1][0] + S_i + w)  &          i > 0 \text{ and } j = 1\\
			\end{array} \right. 
		$$
		Final result is $\max\{f[n][0],f[n][1]\}$.\par
		The state space is $O(n)$, transition function is $O(1)$. Total time is $O(n)$.
	\end{enumerate}
\item [Problem 5]\textbf{Solution:}\par
Denote $f[i]$ represents the maximum result of among first $i$ characters.
	$$ f[i]=\left\{
		\begin{array}{ll}
		0   &    i = 0\\
		\max \limits_{0 \le j \le i -1}\{f[j] + quality(y_{j+1}y_{j+2}...y_i)\} &           i \ge 1
		\end{array} \right. 
	$$
	Final result is $f[n]$.\par
	The state space is $O(n)$, transition function is $O(1)$. Total time is $O(n)$.
\item [Problem 6]\textbf{Solution:}\par
	Define, $S[i]$ as the sum of first $i$ items' length.
	$$ S[i]=\left\{
		\begin{array}{ll}
		0   &    i = 0\\
		S[i-1] + C[i] &           i \ge 1
		\end{array} \right. 
	$$
	Then, denote $f[i]$, is the minimum summation of square error.\\
	$$ f[i]=\left\{
		\begin{array}{ll}
		0   &    i = 0\\
		\min \limits_{0 \le j \le i -1}\{f[j] + (L - (S[i] - S[j] + i - j - 1))^2\} &           i \ge 1
		\end{array} \right. 
	$$
	Final result is $f[n]$.\par
	The state space is $O(n)$, transition function is $O(n)$. Total time is $O(n^2)$.
\item [Problem 7]\textbf{Solution:}\par
	Define, $M[i]$ as the maximum price from $i^{th}$ day to the last day.
	$$ M[i]=\left\{
		\begin{array}{ll}
		p(n)   &    i = n\\
		\max\{M[i + 1], p(i)\} &           i \le n - 1
		\end{array} \right. 
	$$
	Define, $Day[i]$ as the maximum price date from $i^{th}$ day to the last day.
	$$ Day[i]=\left\{
		\begin{array}{ll}
		n   &    i = n\\
		Day[i + 1]   &    M[i] = M[i+1], i \le n - 1\\
		i & M[i] > M[i+1], i \le n - 1
		\end{array} \right. 
	$$
	Set variable $Ans$ as the maximum profit we can make.\\
	iterator all price. If $M[i] - p(i)$ larger than current answer, the update the answer, and recode $i$ and $Day[i]$ as the optimal pair $(i,j)$.
	The state space is $O(n)$, transition function is $O(1)$. Total time is $O(n)$.
\item [Problem 8]\textbf{Solution:}\par
\begin{enumerate}
	\item the instance is shown below.\par
		\begin{tabular}{|c|c|c|c|c|}
			\hline
			& 1&  2 &  3 &  4   \\
			\hline
			$x_i$ & 2 & 2 & 2 & 2 \\
			\hline
			$f(i)$ & 1 & 1 & 1 & 2 \\
			\hline
		\end{tabular}\par
		Correct answer is $4$, which shoot laser every day. However the algorithm will give $2$ which is just shooting laser in $4^{th}$ day.
	\item
	Denote $dp[i]$ as the maximum number of robot killed in first $i$ days, and we use the laser in $i^th$ day.\\
	$$ dp[i]=\left\{
		\begin{array}{ll}
		0   &    i = 0\\
		\max \limits_{1\le j \le i - 1} \{f[j] + \min\{x_i, f(i - j)\}\} &           i \le n - 1
		\end{array} \right. 
	$$
	Final result is $\max \limits_{1\le i \le n}f[i]$.\par
	The state space is $O(n)$, transition function is $O(n)$. Total time is $O(n^2)$.
\end{enumerate}
\item [Problem 9]\textbf{Solution:}\par
\begin{enumerate}
	\item
	\begin{tabular}{cccccc}
			\hline
			& Day 1&Day 2 &Day 3 &Day 4& Day 5 \\
			\hline
			$x$ & 1001 & 1001 & 1001 & 1001 & 1001 \\
			\hline
			$s$ & 1000 & 4 & 3 & 2 & 1 \\
			\hline
	\end{tabular}\par
	Optimal solution is reboot in Day 2 and Day 4, work on Day 1 ,3 ,5.\par
	\item
	Denote $dp[i][j]$ as the maximum number of terabytes processed $i$ days and computer has been continually working for $j$ days. 
	$$ dp[i][j]=\left\{
		\begin{array}{ll}
		0   &    i = 0\\
		\max \limits_{1\le j \le i} \{f[i-1][j]\} & i \ge 1, j = 0\\
		f[i-1][j-1] + \min\{x_i, s[j]\} &           i \ge 1, j \ge 1
		\end{array} \right. 
	$$.
	Also, when we process the $dp$, we need use $pre[i]$ to records which $j$ update the $dp[i][0], 1 \le i \le n$ i.e. $pre[i] = argmax_{1\le j \le i-1}\{dp[i-1][j]\}$.\par
	Then, first we find the maximum data we processed in $n^{th}$ day and record its corresponding $j$. So from $n - j + 1$ day to $n^{th}$ day, it is works and $n - j$ day is reboot. Then we use $pre[n - j]$ to find how many days computer continually works in this period of time. By using the method shows above again and again until we meet the first day. Now we have construct the optimal solution.
\end{enumerate}
\item [Problem 10]\textbf{Solution:}\par
\begin{enumerate}
	\item The instance is shown below:\par
	\begin{tabular}{ccc}
			\hline
			& Day 1& Day 2 \\
			\hline
			$x$ & 1 & 1000 \\
			\hline
			$s$ & 2 & 1 \\
			\hline
	\end{tabular}\par
	Correct answer is 1001, but the algorithm will find 3.
	\item Denote $f[i][0]$ represents the maximum result of among first $i$ days, and using computer $A$ in $i^{th}$ day. $f[i][1]$ represents the maximum result of among first $i$ days, and using computer $B$ in $i^{th}$ day. Then 

	$$ dp[i][j]=\left\{
		\begin{array}{ll}
		a_1 & i = 1, j = 0\\
		b_1,& i = 1, j = 1\\
		\max  \{f[i-1][0] + a_i, f[i-2][1] + b_i\} & i \ge 2, j = 0\\
		\max  \{f[i-1][1] + b_i, f[i-2][0] + a_i\} & i \ge 2, j = 1\\
		\end{array} \right. 
	$$
	Final result is $\max \{f[n][0],f[n][1]\}$.\par
	The state space is $O(n)$, transition function is $O(1)$. Total time is $O(n)$.
	\end{enumerate}
\end{enumerate}
\end{document}
